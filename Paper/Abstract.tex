\documentclass[a4paper,11pt]{article}

\usepackage[utf8]{inputenc}
\usepackage[T5]{fontenc}
\usepackage[english]{babel}
\usepackage{mathptmx}
\usepackage{geometry}
\geometry{
	a4paper,
	left=30mm,
	right=20mm,
	top=22mm,
	bottom=20mm,
}

\title{\vspace{-2em}\bfseries
	Building an Adaptive Vietnamese License Plate Recognition and Retrieval System using Multi-Task Deep Learning
}

\author{
	Phuoc Minh Hieu Pham*, Sy Sieu Cao\\
	\small \textit{*Corresponding author: hieu.2201700085@st.umt.edu.vn}
}

\date{}

\begin{document}

\maketitle

\begin{abstract}
	Automatic License Plate Recognition (ALPR) are an essential component of intelligent transportation, yet their performance is often significantly degraded by real-world image distortions and regional plate format complexities. This research addresses these challenges by proposing a highly adaptive, multi-task deep learning framework specifically designed for the Vietnamese license plate context. The system targets unique diversity of Vietnamese plates while robustly handling low-quality image inputs.
	
	The proposed framework operates as a multi-stage, conditional pipeline.	First, a real-time object detection model performs the localization of all license plate instances. The system's core component is a lightweight Quality Assessment Module (QAM), which acts as an intelligent router, analyzing and classifying each detected plate as either "clear" or "degraded".
	
	The system's adaptive nature is demonstrated in the subsequent step: "degraded" images are selectively forwarded to a specialized restoration neural network, trained to reverse complex degradation processes such as motion blur, glare, and geometric distortion. "Clear" images bypass this resource-intensive step, thereby optimizing system throughput.
	
	Following this conditional restoration, a robust Optical Character Recognition (OCR) model executes the recognition task, transcribing the plate's character string. Finally, the recognized string is used as a query key to an associated database, enabling vehicle information retrieval.
	
	This multi-task approach—integrating detection, quality assessment, conditional restoration, and recognition—demonstrates significant accuracy improvements under challenging real-world conditions compared to traditional, non-adaptive pipelines. The system provides a robust and efficient solution for practical ALPR and information retrieval applications within the specific context of Vietnam.

\end{abstract}

\vspace{0.5em}

\textbf{\small Keywords—}
License plate recognition, multi-task learning, deep learning, Vietnamese license plate dataset, image retrieval.
\end{document}
